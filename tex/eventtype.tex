\renewcommand*{\arraystretch}{1.3}
\newcommand*{\thline}{\specialrule{.1em}{.05em}{.05em}}
\renewcommand*\thesubsection{\arabic{subsection}}

\titleformat*{\section}{\justify\LARGE\bfseries}
\titleformat*{\subsection}{\justify\Large\bfseries}

\let\oldminipage=\minipage
\let\endoldminipage=\endminipage
\renewenvironment{minipage}{\vspace{-2.5mm}\begin{oldminipage}}{\end{oldminipage}\vspace{2mm}}

\setcounter{section}{-1}

\section{Typographical convention}

\centering\begin{tabularx}{\textwidth}{| L{2.55cm} L{6cm} L{6.3cm} |}
\multicolumn{3}{C{14.85cm}}{\Large \textbf{EventType\_Subtype}}                \\
\specialrule{.1em}{.05em}{.05em} 
\multicolumn{2}{|L{8.55cm}}{
	\begin{minipage}{8.55cm}
		A description of the event type or subtype.
	\end{minipage}
	} & \annexe{Event mention example sentence with the trigger word in \anntrg{bold}.} \\ \thline
ParticipantArgument & Description of the Participant argument. & \annexe{Participant argument example sentence with the argument \exargpart{highlighted}} \\
FILLERARGUMENT & Brief description of FILLER argument. For full description for FILLER arguments see \ref{sec:FILLERtypes}. Note: TIME and PLACE FILLERs are allowed on all event types and are not included in event type overviews. & \annexe{FILLER argument example sentence with the FILLER argument \exargfill{highlighted}.}  \\
\specialrule{.1em}{.05em}{.05em} 
\end{tabularx}

\section{Event types}

%-------CSR/Brand----------
\subsection{CSR/Brand}

\centering\begin{tabularx}{\textwidth}{| L{2.55cm} L{6cm} L{6.3cm} |}
\multicolumn{3}{C{14.85cm}}{\Large \textbf{CSR/Brand}}                \\
\specialrule{.1em}{.05em}{.05em} 
\multicolumn{2}{|L{8.55cm}}{
	\begin{minipage}{8.55cm}
		Corporate Social Responsibility and Branding events take place when the company's effects on environmental and social wellbeing are assessed or when the brand image is affected. Includes environmental and social efforts (e.g. affirmative action, employee well-being) and scandals violating responsibility such as sexism accusations, corruption, bribing, etc. Branding includes events improving or damaging image and credibility of the brand image, company image, or product image. This includes advertising campaigns, affiliations, sponsorships.
	\end{minipage}
	} & \annexe{example}                                                                         \\ \thline
Company & The company or brand directly associated with the CSR/Brand event  & \annexe{Argument 1 example sentence with the argument highlighted} \\
\specialrule{.1em}{.05em}{.05em} 
\end{tabularx}

\begin{itemize}[noitemsep,leftmargin=*]
	\item Keywords: Scandal, Corruption, Social Responsibility, Environmental Responsibility, Company Image, Brand Image, Product Image, Credibility, Award, Sponsorship, Advertising Campaign, Affiliation.
	\item \url{en.wikipedia.org/wiki/Corporate_social_responsibility}
	\item \url{www.investopedia.com/terms/c/corp-social-responsibility.asp}
\end{itemize}

\vspace{0.5cm}

%-------Capital Returns----------
\subsection{Capital Returns}
\centering\begin{tabularx}{\textwidth}{| L{2.55cm} L{6cm} L{6.3cm} |}
\multicolumn{3}{C{14.85cm}}{\Large \textbf{CapitalReturns}}                \\
\specialrule{.1em}{.05em}{.05em} 
\multicolumn{2}{|L{8.55cm}}{
	\begin{minipage}{8.55cm}
		A company returns capital to its shareholders. The typical methods are Stock Buy Backs (or Share Repurchases) and Dividends.
	\end{minipage}
	} & \annexe{example}                                                                         \\ \thline
Returnee & The company giving out dividends or buying back its stock  & \annexe{Argument 1 example sentence with the argument highlighted} \\
Amount & Amount of capital returned to the company in monetary amount, amount of shares bought-back, or dividends payed-out.  & \annexe{Argument 2 example sentence with the argument highlighted}  \\
\specialrule{.1em}{.05em}{.05em}
\end{tabularx}

\begin{itemize}[noitemsep,leftmargin=*]
	\item \url{www.investopedia.com/terms/r/returnofcapital.asp}
	\item \url{www.investopedia.com/terms/b/buyback.asp}
	\item \url{en.wikipedia.org/wiki/Share_repurchase}
	\item \url{corporatefinanceinstitute.com/resources/knowledge/finance/dividend-vs-share-buyback-repurchase/}
	\item \url{https://en.wikipedia.org/wiki/Dividend}
	\item \url{https://www.investopedia.com/terms/d/dividendyield.asp}
\end{itemize}

\vspace{0.5cm}

\centering\begin{tabularx}{\textwidth}{| L{2.55cm} L{6cm} L{6.3cm} |}
\multicolumn{3}{C{14.85cm}}{\Large CapitalReturns\_\textbf{ShareRepurchase}}                \\
\specialrule{.1em}{.05em}{.05em} 
\multicolumn{2}{|L{8.55cm}}{
	\begin{minipage}{8.55cm}
		Share repurchase (or stock buyback) is the re-acquisition by a company of its own stock.
		A share repurchase is a program by which a company buys back its own shares from the marketplace, usually because management thinks the shares are undervalued, and thereby reducing the number of outstanding shares. 
		The company buys shares directly from the market or offers its shareholders the option of tendering their shares directly to the company at a fixed price.
	\end{minipage}
	} & \annexe{example}                                                                         \\ \thline
Returnee & The company giving out dividends or buying back its stock  & \annexe{Argument 1 example sentence with the argument highlighted} \\
Amount & Amount of capital returned to the company in monetary amount, amount of shares bought-back, or dividends payed-out.  & \annexe{Argument 2 example sentence with the argument highlighted}  \\
Price & Unit price of the shares. & \\
\specialrule{.1em}{.05em}{.05em}
\end{tabularx}

\vspace{0.5cm}

\centering\begin{tabularx}{\textwidth}{| L{2.55cm} L{6cm} L{6.3cm} |}
\multicolumn{3}{C{14.85cm}}{\Large CapitalReturns\_\textbf{DividendPayment}}                \\
\specialrule{.1em}{.05em}{.05em} 
\multicolumn{2}{|L{8.55cm}}{
	\begin{minipage}{8.55cm}
		A dividend is a payment made by a corporation to its shareholders, usually as a distribution of profits.
		When a corporation earns a profit or surplus, the corporation is able to re-invest the profit in the business (called retained earnings) and pay a proportion of the profit as a dividend to shareholders.
		Distribution to shareholders may be in cash (usually a deposit into a bank account).
	\end{minipage}
	} & \annexe{example}                                                                         \\ \thline
Returnee & The company giving out dividends or buying back its stock  & \annexe{Argument 1 example sentence with the argument highlighted} \\
Amount & Amount of capital returned to the company in monetary amount, amount of shares bought-back, or dividends payed-out.  & \annexe{Argument 2 example sentence with the argument highlighted}  \\
YieldRatio & The current dividend yield ratio in percentage. &                                    \\
\specialrule{.1em}{.05em}{.05em} 
\end{tabularx}

\vspace{0.5cm}

\centering\begin{tabularx}{\textwidth}{| L{2.55cm} L{6cm} L{6.3cm} |}
\multicolumn{3}{C{14.85cm}}{\Large CapitalReturns\_\textbf{DividendYieldRaise}}                \\
\specialrule{.1em}{.05em}{.05em} 
\multicolumn{2}{|L{8.55cm}}{
	\begin{minipage}{8.55cm}
		The dividend yield ratio has increased over a period and has changed substantially as a positive percentage increase compared to a historical trend.
		The dividend yield is a financial ratio that indicates how much a company pays out in dividends each year relative to its share price.
		Dividend yield is represented as a percentage and can be calculated by dividing the dollar value of dividends paid in a given year per share of stock held by the dollar value of one share of stock.
	\end{minipage}
	} & \annexe{example}                                                                         \\ \thline
Returnee & The company giving out dividends or buying back its stock  & \annexe{Argument 1 example sentence with the argument highlighted} \\
Amount & Capital returned to the company in gross monetary amount, amount of shares bought-back, or dividends payed-out.  & \annexe{Argument 2 example sentence with the argument highlighted}  \\
YieldRatio & The current dividend yield ratio in percentage. &                                    \\
HistoricalYieldRatio & The dividend yield ratio in percentage compared to which the current yield ratio has increased. &        \\
\specialrule{.1em}{.05em}{.05em} 
\end{tabularx}

\vspace{0.5cm}

\centering\begin{tabularx}{\textwidth}{| L{2.55cm} L{6cm} L{6.3cm} |}
\multicolumn{3}{C{14.85cm}}{\Large CapitalReturns\_\textbf{DividendYieldReduction}}                \\
\specialrule{.1em}{.05em}{.05em} 
\multicolumn{2}{|L{8.55cm}}{
	\begin{minipage}{8.55cm}
		The dividend yield ratio has decreased over a period and has changed substantially as a negative percentage decrease compared to a historical trend.
		The dividend yield is a financial ratio that indicates how much a company pays out in dividends each year relative to its share price.
		Dividend yield is represented as a percentage and can be calculated by dividing the dollar value of dividends paid in a given year per share of stock held by the dollar value of one share of stock.
	\end{minipage}
	} & \annexe{example}                                                                         \\ \thline
Returnee & The company giving out dividends or buying back its stock  & \annexe{Argument 1 example sentence with the argument highlighted} \\
Amount & Capital returned to the company in monetary amount, amount of shares bought-back, or dividends payed-out.  & \annexe{Argument 2 example sentence with the argument highlighted}  \\
YieldRatio & The current dividend yield ratio in percentage. &                                    \\
HistoricalYieldRatio & The dividend yield ratio in percentage compared to which the current yield ratio has decreased. &        \\
 & & \\
\specialrule{.1em}{.05em}{.05em} 
\end{tabularx}

\vspace{0.5cm}

\centering\begin{tabularx}{\textwidth}{| L{2.55cm} L{6cm} L{6.3cm} |}
\multicolumn{3}{C{14.85cm}}{\Large CapitalReturns\_\textbf{DividendYieldStable}}                \\
\specialrule{.1em}{.05em}{.05em} 
\multicolumn{2}{|L{8.55cm}}{
	\begin{minipage}{8.55cm}
		The dividend yield ratio has remained stable over a period and has not changed substantially compared to a historical trend.
		The dividend yield is a financial ratio that indicates how much a company pays out in dividends each year relative to its share price.
		Dividend yield is represented as a percentage and can be calculated by dividing the dollar value of dividends paid in a given year per share of stock held by the dollar value of one share of stock. 
	\end{minipage}
	} & \annexe{example}                                                                         \\ \thline
Returnee & The company giving out dividends or buying back its stock  & \annexe{Argument 1 example sentence with the argument highlighted} \\
Amount & Capital returned to the company in monetary amount, amount of shares bought-back, or dividends payed-out.  & \annexe{Argument 2 example sentence with the argument highlighted}  \\
YieldRatio & The current dividend yield ratio in percentage. &                                    \\
HistoricalYieldRatio & The dividend yield ratio in percentage compared to which the current yield ratio has remained stable. &        \\
\specialrule{.1em}{.05em}{.05em} 
\end{tabularx}

\vspace{0.5cm}

%-------Deal---------
\subsection{Deal}

\centering\begin{tabularx}{\textwidth}{| L{2.55cm} L{6cm} L{6.3cm} |}
\multicolumn{3}{C{14.85cm}}{\Large \textbf{Deal}}                \\
\specialrule{.1em}{.05em}{.05em} 
\multicolumn{2}{|L{8.55cm}}{
	\begin{minipage}{8.55cm}
		Deals and partnerships to cooperate with another company or entity. Service and product deals, licensing, contract bid, alliance, partnership, Memorandum of Understanding (MOU), pacts, joint ventures (two companies pool resources to accomplish a task), collaborations, contracts, agreements, development partnerships (usually public private partnerships for development projects), and affiliations.
	\end{minipage}
	} & \annexe{example}                                                                         \\ \thline
Partner & One of the partner in the deal. A deal by definition has multiple partners. & \annexe{Argument 1 example sentence with the argument highlighted} \\
Goal & Goal and aims of the deal. & \\

\specialrule{.1em}{.05em}{.05em} 
\end{tabularx}
Keywords: Service and product deals, licensing, contract bid, alliance, partnership, Memorandum of Understanding (MOU), pacts, joint ventures (two companies pool resources to accomplish a task), collaborations, contracts, agreements, development partnerships (usually public private partnerships for development projects), and affiliations.
\begin{itemize}
	\item \url{en.wikipedia.org/wiki/Corporate_social_responsibility}
\end{itemize}

\vspace{0.5cm}

%-------Employment---------
\subsection{Employment}

\centering\begin{tabularx}{\textwidth}{| L{2.55cm} L{6cm} L{6.3cm} |}
\multicolumn{3}{C{14.85cm}}{\Large \textbf{Employment}}                \\
\specialrule{.1em}{.05em}{.05em} 
\multicolumn{2}{|L{8.55cm}}{
	\begin{minipage}{8.55cm}
		Events regarding employment changes (firing and hiring), compensation, and issues regarding the workforce or other employees such as executives.
		Includes CEO changes, executive change, board change, executive compensation, employment issues, strikes, workforce increase, workforce decrease/firing.
	\end{minipage}
	} & \annexe{example}                                                                         \\ \thline
Employee & Employee, group of employees, part of the workforce. & \annexe{Argument 2 example sentence with the argument highlighted}  \\
Employer & Employer company or employing person.  & \annexe{Argument example sentence with the argument highlighted} \\
TITLE & The job title/position of the employee. & \annexe{FILLER example sentence with the argument highlighted} \\
\specialrule{.1em}{.05em}{.05em} 
\end{tabularx}

\vspace{0.5cm}

\centering\begin{tabularx}{\textwidth}{| L{2.55cm} L{6cm} L{6.3cm} |}
\multicolumn{3}{C{14.85cm}}{\Large Employment\_\textbf{Start}}                \\
\specialrule{.1em}{.05em}{.05em}
\multicolumn{2}{|L{8.55cm}}{
	\begin{minipage}{8.55cm}
		A person or group of people start a new job position at an employer (company or organization). Includes events, announcements, speculation, expectation, changes on hiring, CEO changes, executive change, board change, workforce increases.
	\end{minipage}
	} & \annexe{example}                                                                         \\ \thline
Employee & Employee, group of employees, part of the workforce starting a new position. & \annexe{Argument 2 example sentence with the argument highlighted}  \\
Employer & Employer company or employing person.  & \annexe{Argument example sentence with the argument highlighted} \\
Replacing & The person that the starting Employee replaces. & \annexe{Argument example sentence with the argument highlighted}\\
TITLE & The job title/position of the starting Employee. & \annexe{Argument example sentence with the argument highlighted} \\
\specialrule{.1em}{.05em}{.05em} 
\end{tabularx}

\vspace{0.5cm}

\centering\begin{tabularx}{\textwidth}{| L{2.55cm} L{6cm} L{6.3cm} |}
\multicolumn{3}{C{14.85cm}}{\Large Employment\_\textbf{End}}                \\
\specialrule{.1em}{.05em}{.05em} 
\multicolumn{2}{|L{8.55cm}}{
	\begin{minipage}{8.55cm}
		The job position of a person or a group of people has ended with an employer (company or organization). Includes events, announcements, speculation, expectation, changes CEO change, executive change, board change, workforce decrease/firing, termination, death, illness, and any other form of ending a professional position.
	\end{minipage}
	} & \annexe{example}                                                                         \\ \thline
Employee & Employee, group of employees, part of the workforce being terminated/fired. & \annexe{Argument 2 example sentence with the argument highlighted}  \\
Employer & Employer company or employing person doing the firing.  & \annexe{Argument example sentence with the argument highlighted} \\
Replacer & The person that replaces the terminated Employee. & \annexe{Argument example sentence with the argument highlighted}\\
TITLE & The job title/position of the terminated Employee. & \annexe{Argument example sentence with the argument highlighted} \\

\specialrule{.1em}{.05em}{.05em} 
\end{tabularx}

\vspace{0.5cm}

\centering\begin{tabularx}{\textwidth}{| L{2.55cm} L{6cm} L{6.3cm} |}
\multicolumn{3}{C{14.85cm}}{\Large Employment\_\textbf{Compensation}}                \\
\specialrule{.1em}{.05em}{.05em} 
\multicolumn{2}{|L{8.55cm}}{
	\begin{minipage}{8.55cm}
		Changes in compensation for employees. Executive compensation as equity, workforce pay-out, payment issues, or wage in/decreases.
	\end{minipage}
	} & \annexe{example}                                                                         \\ \thline
Employee & Employee, group of employees, part of the workforce who's compensation is affected. & \annexe{Argument 2 example sentence with the argument highlighted}  \\
Employer & Employer company or employing person.  & \annexe{Argument example sentence with the argument highlighted} \\
Amount & the monetary amount of compensation. & \\
TITLE & The job title/position of the terminated Employee. & \annexe{Argument example sentence with the argument highlighted} \\
\specialrule{.1em}{.05em}{.05em} 
\end{tabularx}

\vspace{0.5cm}

\centering\begin{tabularx}{\textwidth}{| L{2.55cm} L{6cm} L{6.3cm} |}
\multicolumn{3}{C{14.85cm}}{\Large Employment\_\textbf{Problem}}                \\
\specialrule{.1em}{.05em}{.05em} 
\multicolumn{2}{|L{8.55cm}}{
	\begin{minipage}{8.55cm}
		Issues, strikes, hiring failure, and other problems regarding employment.
	\end{minipage}
	} & \annexe{example}                                                                         \\ \thline
Employee & Employee, group of employees, part of the workforce who is involved in a problem. & \annexe{Argument 2 example sentence with the argument highlighted}  \\
Employer & Employer company or employing person.  & \annexe{Argument example sentence with the argument highlighted} \\
TITLE & The job title/position of the terminated Employee. & \annexe{Argument example sentence with the argument highlighted} \\
\specialrule{.1em}{.05em}{.05em} 
\end{tabularx}
Note: has overlap with CSR/Brand.

\vspace{0.5cm}

%-------Facility---------
\subsection{Facility}

\centering\begin{tabularx}{\textwidth}{| L{2.55cm} L{6cm} L{6.3cm} |}
\multicolumn{3}{C{14.85cm}}{\Large \textbf{Facility}}                \\
\specialrule{.1em}{.05em}{.05em} 
\multicolumn{2}{|L{8.55cm}}{
	\begin{minipage}{8.55cm}
		Events pertaining to a company's physical presence as facilities, factories, headquarters, warehouses, and other real-estate. Facilities opening, facilities closing, headquarters relocation, headquarters opening, headquarters closing. Facilities include headquarters, retail sites, production sites, logic centers, etc.
	\end{minipage}
	} & \annexe{example}                                                                         \\ \thline
Facility & Facility, factory or headquarters. & \annexe{Argument 2 example sentence with the argument highlighted}  \\
Company & Company that is owner of the facility or headquarters.  & \annexe{Argument 1 example sentence with the argument highlighted} \\

\specialrule{.1em}{.05em}{.05em} 
\end{tabularx}

\vspace{0.5cm}

\centering\begin{tabularx}{\textwidth}{| L{2.55cm} L{6cm} L{6.3cm} |}
\multicolumn{3}{C{14.85cm}}{\Large Facility\_\textbf{Open}}                \\
\specialrule{.1em}{.05em}{.05em} 
\multicolumn{2}{|L{8.55cm}}{
	\begin{minipage}{8.55cm}
		Facilities opening, includes headquarters, factories, production sites, retail sites, etc.
	\end{minipage}
	} & \annexe{example}                                                                         \\ \thline
Facility & Facility or headquarters & \annexe{Argument 2 example sentence with the argument highlighted}  \\
Company & Company opening facility or headquarters.  & \annexe{Argument 1 example sentence with the argument highlighted} \\

\specialrule{.1em}{.05em}{.05em} 
\end{tabularx}

\vspace{0.5cm}

\centering\begin{tabularx}{\textwidth}{| L{2.55cm} L{6cm} L{6.3cm} |}
\multicolumn{3}{C{14.85cm}}{\Large Facility\_\textbf{Close}}                \\
\specialrule{.1em}{.05em}{.05em} 
\multicolumn{2}{|L{8.55cm}}{
	\begin{minipage}{8.55cm}
		Facilities closing, includes headquarters, factories, production sites, retail sites, etc.
	\end{minipage}
	} & \annexe{example}                                                                         \\ \thline
Facility & Facility or headquarters & \annexe{Argument 2 example sentence with the argument highlighted}  \\
Company & Company opening facility or headquarters.  & \annexe{Argument 1 example sentence with the argument highlighted} \\
\specialrule{.1em}{.05em}{.05em} 
\end{tabularx}

\vspace{0.5cm}

%-------FinancialResult---------
\subsection{Financial Result}

\centering\begin{tabularx}{\textwidth}{| L{2.55cm} L{6cm} L{6.3cm} |}
\multicolumn{3}{C{14.85cm}}{\Large \textbf{FinancialResult}}                \\
\specialrule{.1em}{.05em}{.05em} 
\multicolumn{2}{|L{8.55cm}}{
	\begin{minipage}{8.55cm}
		Financial results as discussed in reports published by the company or other instance.
		Quarterly and annual reports include key accounting and financial data for a company, including gross revenue, net profit, operational expenses and cash flow.
		A quarterly report for a public company typically includes an income statement, balance sheet, and cash flow statement for the quarter and the year-to-date (YTD), as well as comparative results for the prior year.
		This event captures publishing, announcement or discussions of these reports and reported results within them.\\
		Overlap with \type{Financing}, \type{Investment}, \type{Profit/Loss}, \type{Revenue}, \type{Sales Volume}: \type{FinancialResult}
		Result contained in these reports should be tagged into these more specific event types.
		\type{FinancialResult} is more general than these types and should be tagged when talking about a report or when the result does not fit in the more specific categories.
	\end{minipage}
	} & \annexe{example}                                                                         \\ \thline
Reportee & Company that is the topic of the financial report. & \\
Result & The type of result with associated monetary amount that is reported on. & \\
\specialrule{.1em}{.05em}{.05em} 
\end{tabularx}

\vspace{0.5cm}

\centering\begin{tabularx}{\textwidth}{| L{2.55cm} L{6cm} L{6.3cm} |}

\multicolumn{3}{C{14.85cm}}{\Large FinancialResult\_\textbf{Beat}}                \\
\specialrule{.1em}{.05em}{.05em} 
\multicolumn{2}{|L{8.55cm}}{
	\begin{minipage}{8.55cm}
		A financial report shows that a company's financials are better than previous expectations or historical trends.
	\end{minipage}
	} & \annexe{example}                                                                         \\ \thline
Reportee & Company that is the topic of the financial report. & \\
Result & The type of result with associated monetary amount that is reported on. & \\
\specialrule{.1em}{.05em}{.05em} 
\end{tabularx}

\vspace{0.5cm}

\centering\begin{tabularx}{\textwidth}{| L{2.55cm} L{6cm} L{6.3cm} |}

\multicolumn{3}{C{14.85cm}}{\Large FinancialResult\_\textbf{Miss}}                \\
\specialrule{.1em}{.05em}{.05em} 
\multicolumn{2}{|L{8.55cm}}{
	\begin{minipage}{8.55cm}
		A financial report shows that a company's financials are worse than previous expectations or historical trends.
	\end{minipage}
	} & \annexe{example}                                                                         \\ \thline
Reportee & Company that is the topic of the financial report. & \\
Result & The type of result with associated monetary amount that is reported on. & \\
\specialrule{.1em}{.05em}{.05em} 
\end{tabularx}

\vspace{0.5cm}

\centering\begin{tabularx}{\textwidth}{| L{2.55cm} L{6cm} L{6.3cm} |}

\multicolumn{3}{C{14.85cm}}{\Large FinancialResult\_\textbf{Stable}}                \\
\specialrule{.1em}{.05em}{.05em} 
\multicolumn{2}{|L{8.55cm}}{
	\begin{minipage}{8.55cm}
		A financial report shows that a company's financials are stable or as expected w.r.t. previous expectations or historical trends.
	\end{minipage}
	} & \annexe{example}                                                                         \\ \thline
Reportee & Company that is the topic of the financial report. & \\
Result & The type of result with associated monetary amount that is reported on. & \\
\specialrule{.1em}{.05em}{.05em} 
\end{tabularx}

\vspace{0.5cm}

%-------Financing---------
\subsection{Financing}

\centering\begin{tabularx}{\textwidth}{| L{2.55cm} L{6cm} L{6.3cm} |}
\multicolumn{3}{C{14.85cm}}{\Large \textbf{Financing}}                \\
\specialrule{.1em}{.05em}{.05em} 
\multicolumn{2}{|L{8.55cm}}{
	\begin{minipage}{8.55cm}
		Financing of a company are ways in which it raises capital through debt and equity.
		Equity financing involves not just the sale of stocks, but also the sale of other equity or quasi-equity instruments such as preferred stock, convertible preferred stock and equity units that include common shares and warrants.
		These events focus on how a company raises equity through issuing shares in IPOs, stock splits, etc.
		Debt Financing refers to the way in which companies raise funds by taking on debt by issuing bonds or taking on loans. Debt financing occurs when a firm sells fixed income products, such as bonds, bills, or notes, to investors.
		Event mentions pertaining to company debt and debt ratios. Includes debt announcements, debt forecasts, debt increases, debt reductions, and debt and equity restructuring.
		The balance between equity and debt of a company is expressed as the weighted average cost of capital, or WACC.
		\\
		Overlap with \type{Ratings\_Debt} and \type{Investment}.
		Difference with \type{Ratings\_Debt} that this is not a rating by an analyst. \type{Financing} events are focused around a company taking on debt to raise capital.
		Difference with \type{Investment} is that \type{Financing} events are focused around means of raising capital and not the investor-investee relation in se.
	\end{minipage}
	} & \annexe{example}                                                                         \\ \thline
Financee & The company selling equity as shares or taking on debt by issuing bonds or taking out loans. & \\
Financer & The company, institution, or private entity investing in equity or giving out the loan. & \\
Amount & Amount of capital financed as debt or equity. & \\
\specialrule{.1em}{.05em}{.05em} 
\end{tabularx}

\begin{itemize}[noitemsep,leftmargin=*]
    \item \url{https://www.investopedia.com/terms/d/debtfinancing.asp}
    \item \url{https://www.investopedia.com/terms/e/equityfinancing.asp}
    \item \url{https://www.investopedia.com/ask/answers/032515/how-does-company-choose-between-debt-and-equity-its-capital-structure.asp}
    \item \url{https://www.investopedia.com/financial-edge/1112/small-business-financing-debt-or-equity.aspx}
    \item \url{https://finance.zacks.com/differences-between-debt-equity-investments-3035.html}
    \item \url{https://corporatefinanceinstitute.com/resources/knowledge/finance/debt-vs-equity/}
    \item \url{https://www.investopedia.com/terms/b/bond.asp}
\end{itemize}

\vspace{0.5cm}

%-------Investment---------
\subsection{Investment}

\centering\begin{tabularx}{\textwidth}{| L{2.55cm} L{6cm} L{6.3cm} |}
\multicolumn{3}{C{14.85cm}}{\Large \textbf{Investment}}                \\
\specialrule{.1em}{.05em}{.05em} 
\multicolumn{2}{|L{8.55cm}}{
	\begin{minipage}{8.55cm}
		Investment in other companies or subsidiaries.
		This event involves one company investing in another company it owns as a subsidiary or does not control or own fully as in affiliate/associate companies.
		Corporate investments are often Capital Investments as part of a Corporate Finance strategy.
		This event has overlap with i) MergerAcquisition and ii) Deal: i) the difference between MergerAcquisition and Investment is that in the MergerAcquisition event one company gains full ownership of another company or parts of that company. This is not the case in Investment where partial ownership through share investment is possible but not the goal. ii) Deal is more general than Investment, an Investment is between two companies with the investing party investing some capital expecting a return. A Deal can be between any two entities and is much more general regarding its meaning.
	\end{minipage}
	} & \annexe{example}                                                                         \\ \thline
Investor & Company or organization that makes the investment. & \\
Investee & Company being in which the Investor invests. Can also be a stock or other security. & \\
CapitalInvested & Amount of capital invested as a monetary expression. & \\
\specialrule{.1em}{.05em}{.05em} 
\end{tabularx}

\begin{itemize}[noitemsep,leftmargin=*]
	\item \url{https://www.investopedia.com/terms/c/capital-investment.asp}
	\item cf. Capital Investment \url{https://www.investopedia.com/terms/c/corporatefinance.asp}
\end{itemize}

\vspace{0.5cm}

%-------Legal---------
\subsection{Legal}

\centering\begin{tabularx}{\textwidth}{| L{2.55cm} L{6cm} L{6.3cm} |}
\multicolumn{3}{C{14.85cm}}{\Large \textbf{Legal}}                \\
\specialrule{.1em}{.05em}{.05em} 
\multicolumn{2}{|L{8.55cm}}{
	\begin{minipage}{8.55cm}
		Events pertaining to governmental justice. Investigations. accusations, litigations, arrest charges, Fraud, Money Laundering, Bribing, Settlement, Judgement, Lawsuit, Legal issues, Regulatory issues, injunctions.
	\end{minipage}
	} & \annexe{example}
\\ \thline
Defendant & Entity, company, or person being alleged of a crime, under investigation, convicted, fined, etc. & \\
ALLEGATION & Allegation or other formal/legal offence that is alleged against a Defendant. & \\
\specialrule{.1em}{.05em}{.05em} 
\end{tabularx}

\vspace{0.5cm}

\centering\begin{tabularx}{\textwidth}{| L{2.55cm} L{6cm} L{6.3cm} |}
\multicolumn{3}{C{14.85cm}}{\Large Legal\_\textbf{Proceeding}}                \\
\specialrule{.1em}{.05em}{.05em} 
\multicolumn{2}{|L{8.55cm}}{
	\begin{minipage}{8.55cm}
		A court proceeding has been initiated for a defendant accused of an allegation. Legal proceedings include trials, lawsuits, and hearings. Trials are state-initiated proceedings in a court. Hearing are state-initiated proceedings outside the court (e.g. in Senate). Lawsuits are proceedings between private parties. This also includes charges and indictments, i.e. formal accusations, and investigations.
	\end{minipage}
	} & \annexe{example}                                                                         \\ \thline
Defendant & Entity being convicted, fined, etc. & \\
Complainant & Entity filing suit or initiating the proceeding by accusation of the defendant. Governmental prosecutor, plaintiff, litigator, suer, petitioner. & \\
Adjudicator & Judge or court. In case of investigation, the investigator either private by attorney or public police investigators. & \\
ALLEGATION & Allegation or other formal/legal offence that is alleged against a Defendant. & \\
\specialrule{.1em}{.05em}{.05em} 
\end{tabularx}

\vspace{0.5cm}

\centering\begin{tabularx}{\textwidth}{| L{2.55cm} L{6cm} L{6.3cm} |}
\multicolumn{3}{C{14.85cm}}{\Large Legal\_\textbf{Conviction/Settlement}}                \\
\specialrule{.1em}{.05em}{.05em} 
\multicolumn{2}{|L{8.55cm}}{
	\begin{minipage}{8.55cm}
		A defendant has been found guilty of an allegation and is at the conclusion of a trial.
		The defendant is convicted to a sentence which can be jail-time, a fine, cease-and-desist, etc.
		Also applies where allegations are settled out of court.
	\end{minipage}
	} & \annexe{example}                                                                         \\ \thline
Defendant & Entity being convicted, fined, etc. & \\
Adjudicator & Entity proclaiming the conviction & \\
SENTENCE & Sentence, fine, incarceration, or other punishment. & \\
ALLEGATION & Allegation or other formal/legal offence that is alleged against a Defendant. & \\

\specialrule{.1em}{.05em}{.05em} 
\end{tabularx}

\vspace{0.5cm}

\centering\begin{tabularx}{\textwidth}{| L{2.55cm} L{6cm} L{6.3cm} |}
\multicolumn{3}{C{14.85cm}}{\Large Legal\_\textbf{Acquit}}                \\
\specialrule{.1em}{.05em}{.05em} 
\multicolumn{2}{|L{8.55cm}}{
	\begin{minipage}{8.55cm}
		A defendant has been found not guilty of an allegation at the conclusion of a trial. This is the opposite of conviction.
	\end{minipage}
	} & \annexe{example}                                                                         \\ \thline
Defendant & Entity being convicted, fined, etc. & \\
Adjudicator & Entity proclaiming the conviction. & \\
ALLEGATION & Allegation or other formal/legal offence that is alleged against a Defendant. & \\
\specialrule{.1em}{.05em}{.05em}
\end{tabularx}

\vspace{0.5cm}

\centering\begin{tabularx}{\textwidth}{| L{2.55cm} L{6cm} L{6.3cm} |}
\multicolumn{3}{C{14.85cm}}{\Large Legal\_\textbf{Appeal}}                \\
\specialrule{.1em}{.05em}{.05em} 
\multicolumn{2}{|L{8.55cm}}{
	\begin{minipage}{8.55cm}
		A court decision is taken to a higher court for review.
	\end{minipage}
	} & \annexe{example}                                                                         \\ \thline
Defendant & Entity being convicted, fined, etc. & \\
Complainant & Entity filing suit or initiating the proceeding by accusation of the defendant. Governmental prosecutor, plaintiff, litigator, sue-er, petitioner. & \\
Adjudicator & Entity proclaiming the conviction. & \\
ALLEGATION & Allegation or other formal/legal offence that is alleged against a Defendant. & \\
\specialrule{.1em}{.05em}{.05em} 
\end{tabularx}

\vspace{0.5cm}

%-------Macroeconomics---------
\subsection{Macroeconomics}

\centering\begin{tabularx}{\textwidth}{| L{2.55cm} L{6cm} L{6.3cm} |}
\multicolumn{3}{C{14.85cm}}{\Large \textbf{Macroeconomics}}                \\
\specialrule{.1em}{.05em}{.05em} 
\multicolumn{2}{|L{8.55cm}}{
	\begin{minipage}{8.55cm}
		Macro-economic events are aggregated indicators such as sectorial trends, policy, GDP, unemployment rates, national income, price indices, and the interrelations among the different markets and sectors of the economy. Includes events, speculation, expectation, forecasts, announcements, changes on a company's position in the marker, the state of the market, sector or country in general,  headwinds, tailwinds, business trends, market share, consumer spending, etc.
	\end{minipage}
	} & \annexe{example}                                                                         \\ \thline
AffectedCompany & Company, organization, or entity being affected by the macroeconomic event. & \\
Sector & Sector, industry, market, state, geographical location where the event takes effect. & \\
\specialrule{.1em}{.05em}{.05em} 
\end{tabularx}

\vspace{0.5cm}


%-------Merger/Acquisition---------
\subsection{Merger/Acquisition}

\centering\begin{tabularx}{\textwidth}{| L{2.55cm} L{6cm} L{6.3cm} |}
\multicolumn{3}{C{14.85cm}}{\Large \textbf{Merger/Acquisition}}                \\
\specialrule{.1em}{.05em}{.05em} 
\multicolumn{2}{|L{8.55cm}}{
	\begin{minipage}{8.55cm}
		Consolidation of companies and assets involving at least two companies. A merger is a legal consolidation of two entities into one entity. An acquisition occurs when one entity takes ownership of another entity's stock, equity interests or assets.
	\end{minipage}
	} & \annexe{example}                                                                         \\ \thline
Acquirer & Entity receiving the company or assets & \\
Target & Entity filing suit or initiating the proceeding by accusation of the defendant. Governmental prosecutor, plaintiff, litigator, suer, petitioner. & \\
Cost & Cost of the merger or acquisition in monetary amount. & \\
\specialrule{.1em}{.05em}{.05em} 
\end{tabularx}

\vspace{0.5cm}


%-------Product/Service---------
\subsection{Product/Service}

\centering\begin{tabularx}{\textwidth}{| L{2.55cm} L{6cm} L{6.3cm} |}
\multicolumn{3}{C{14.85cm}}{\Large \textbf{Product/Service}}                \\
\specialrule{.1em}{.05em}{.05em} 
\multicolumn{2}{|L{8.55cm}}{
	\begin{minipage}{8.55cm}
		Events pertaining to a product or service. Includes announcements, launches, testing, changes, up/downgrades, updates, recalls, trial results, approvals, reviews, and other issues regarding products and services of a company.
	\end{minipage}
	} & \annexe{example}                                                                         \\ \thline
ProductService & The Product/Service in question. & \\
Producer & Company, organization, or entity that produces, delivers, or otherwise brings to market the product or service. & \\
\specialrule{.1em}{.05em}{.05em} 
\end{tabularx}

\vspace{0.5cm}

\centering\begin{tabularx}{\textwidth}{| L{2.55cm} L{6cm} L{6.3cm} |}
\multicolumn{3}{C{14.85cm}}{\Large Product/Service\_\textbf{Launch}}                \\
\specialrule{.1em}{.05em}{.05em} 
\multicolumn{2}{|L{8.55cm}}{
	\begin{minipage}{8.55cm}
		A company brings a new product/service to market. Also applicable to product and service updates.
	\end{minipage}
	} & \annexe{example}                                                                         \\ \thline
ProductService & Product/Service being launched. & \\
Producer & Company, organization, or entity that launches (or updates) a new product or service. & \\
\specialrule{.1em}{.05em}{.05em} 
\end{tabularx}

\vspace{0.5cm}

\centering\begin{tabularx}{\textwidth}{| L{2.55cm} L{6cm} L{6.3cm} |}
\multicolumn{3}{C{14.85cm}}{\Large Product/Service\_\textbf{Cancellation/Recall}}                \\
\specialrule{.1em}{.05em}{.05em} 
\multicolumn{2}{|L{8.55cm}}{
	\begin{minipage}{8.55cm}
		A company recalls a product or cancels a product/service.
	\end{minipage}
	} & \annexe{example}                                                                         \\ \thline
ProductService & Product/service being launched & \\
Producer & Company, organization, or entity that recalls or cancels a product or service. & \\
\specialrule{.1em}{.05em}{.05em} 
\end{tabularx}

\vspace{0.5cm}

\centering\begin{tabularx}{\textwidth}{| L{2.55cm} L{6cm} L{6.3cm} |}
\multicolumn{3}{C{14.85cm}}{\Large Product/Service\_\textbf{Trial}}                \\
\specialrule{.1em}{.05em}{.05em} 
\multicolumn{2}{|L{8.55cm}}{
	\begin{minipage}{8.55cm}
		A product or service is being tested or trialed either by a company or regulatory body.\\
		This includes testing runs of prototypes, and approval or disapproval after trial by regulatory agents or governments.
		This type is typical for reporting on bio-technology and pharmaceutical industries which require trial by a regulatory body. Technology and consumers products/services are also often trialed before being brought to market.
	\end{minipage}
	} & \annexe{example} \\ \thline
ProductService & Product/service being launched & \\
Producer & Company, organization, or entity that is testing the product/service. & \\
Trialer & Governmental or private regulator, organization, or other entity overseeing the trial. & \\                    
\specialrule{.1em}{.05em}{.05em}
\end{tabularx}

\vspace{0.5cm}

%-------Profit/Loss---------
\subsection{Profit/Loss}

\centering\begin{tabularx}{\textwidth}{| L{2.55cm} L{6cm} L{6.3cm} |}
\multicolumn{3}{C{14.85cm}}{\Large \textbf{Profit/Loss}}                \\
\specialrule{.1em}{.05em}{.05em} 
\multicolumn{2}{|L{8.55cm}}{
	\begin{minipage}{8.55cm}
		Profits are financial benefits that are realized when the amount of revenue exceeds expenses. Inversely, losses happen when the expense exceeds the revenue. We include declarations and forecasts of profit/loss, positive and negative (losses) profit, lower than, higher than, as expected, increased, decreased, and stable profits. Has conceptual overlap with Revenue.
	\end{minipage}
	} & \annexe{example} \\ \thline
Profiteer & Company, organization, or entity turning a profit or loss. & \\
Amount & The amount of money won/lost. & \\
\specialrule{.1em}{.05em}{.05em} 
\end{tabularx}

\vspace{0.5cm}

\centering\begin{tabularx}{\textwidth}{| L{2.55cm} L{6cm} L{6.3cm} |}
\multicolumn{3}{C{14.85cm}}{\Large Profit/Loss\_\textbf{Increase}}                \\
\specialrule{.1em}{.05em}{.05em} 
\multicolumn{2}{|L{8.55cm}}{
	\begin{minipage}{8.55cm}
        Profit or Loss has increased compared to a historical trend. We do not differentiate between Profit and Loss: this event applies to whichever is the topic of topic of the event.\\
		Overlap with \type{Revenue}:
	\end{minipage}
	} & \annexe{example}    \\ \thline
Profiteer & Company, organization, or entity turning a profit or loss. & \\
Amount & The current amount of money won/lost. & \\
HistoricalAmount & The historical amount of money won or lost compared to which the current amount is higher. & \\
\specialrule{.1em}{.05em}{.05em} 
\end{tabularx}

\vspace{0.5cm}

\centering\begin{tabularx}{\textwidth}{| L{2.55cm} L{6cm} L{6.3cm} |}
\multicolumn{3}{C{14.85cm}}{\Large Profit/Loss\_\textbf{Decrease}}                \\
\specialrule{.1em}{.05em}{.05em} 
\multicolumn{2}{|L{8.55cm}}{
	\begin{minipage}{8.55cm}
        Profit or Loss has decreased compared to a historical trend. We do not differentiate between Profit and Loss: this event applies to whichever is the topic of topic of the event.\\
		Overlap with \type{Revenue}:
	\end{minipage}
	} & \annexe{example}       \\ \thline
Profiteer & Company, organization, or entity turning a profit or loss. & \\
Amount & The current amount of money won/lost. & \\
HistoricalAmount & The historical amount of money won or lost compared to which the current amount is lower. & \\
\specialrule{.1em}{.05em}{.05em} 
\end{tabularx}

\vspace{0.5cm}

\centering\begin{tabularx}{\textwidth}{| L{2.55cm} L{6cm} L{6.3cm} |}
\multicolumn{3}{C{14.85cm}}{\Large Profit/Loss\_\textbf{Decrease}}                \\
\specialrule{.1em}{.05em}{.05em} 
\multicolumn{2}{|L{8.55cm}}{
	\begin{minipage}{8.55cm}
        Profit or Loss has decreased compared to a historical trend. We do not differentiate between Profit and Loss: this event applies to whichever is the topic of topic of the event.\\
		Overlap with \type{Revenue}:
	\end{minipage}
	} & \annexe{example}                                                                         \\ \thline
Profiteer & Company, organization, or entity turning a profit or loss. & \\
Amount & The current amount of money won/lost. & \\
HistoricalAmount & The historical amount of money won or lost compared to which the current amount is lower. & \\
\specialrule{.1em}{.05em}{.05em} 
\end{tabularx}

\vspace{0.5cm}

\centering\begin{tabularx}{\textwidth}{| L{2.55cm} L{6cm} L{6.3cm} |}
\multicolumn{3}{C{14.85cm}}{\Large Profit/Loss\_\textbf{Stable}}                \\
\specialrule{.1em}{.05em}{.05em} 
\multicolumn{2}{|L{8.55cm}}{
	\begin{minipage}{8.55cm}
        Profit or Loss has remained stable or there is no significant change compared to a historical trend. We do not differentiate between Profit and Loss: this event applies to whichever is the topic of topic of the event.\\
		Overlap with \type{Revenue}:
	\end{minipage}
	} & \annexe{example}                                                                         \\ \thline
Profiteer & Company, organization, or entity turning a profit or loss. & \\
Amount & The current amount of money won/lost. & \\
HistoricalAmount & The historical amount of money won or lost compared to which the current amount remained stable. & \\
\specialrule{.1em}{.05em}{.05em} 
\end{tabularx}

\vspace{0.5cm}


%-------Rating---------
\subsection{Rating}

\centering\begin{tabularx}{\textwidth}{| L{2.55cm} L{6cm} L{6.3cm} |}
\multicolumn{3}{C{14.85cm}}{\Large \textbf{Rating}}                \\
\specialrule{.1em}{.05em}{.05em} 
\multicolumn{2}{|L{8.55cm}}{
	\begin{minipage}{8.55cm}
		Analyst ratings and advice on securities such as stocks but also credit and debt ratings. Includes announcements, forecasts, speculation, expectation, reports, etc. on analyst ratings, analyst advice, analysts setting Price targets, analysts credit-debt ratings, changes in rating agency listing, or buy/sell/hold and upgrade/downgrade/maintain advice.
	\end{minipage}
	} & \annexe{example}                                                                         \\ \thline
Security & Security or company's debt being rated. & \\
Analyst & Rater, advisor or analyst publishing the rating. & \\
\specialrule{.1em}{.05em}{.05em} 
\end{tabularx}

\begin{itemize}
    \item \url{https://www.investopedia.com/financial-edge/0512/understanding-analyst-ratings.aspx}
\end{itemize}

\vspace{0.5cm}

\centering\begin{tabularx}{\textwidth}{| L{2.55cm} L{6cm} L{6.3cm} |}
\multicolumn{3}{C{14.85cm}}{\Large Rating\_\textbf{BuyOutperform}}                \\
\specialrule{.1em}{.05em}{.05em} 
\multicolumn{2}{|L{8.55cm}}{
	\begin{minipage}{8.55cm}
		Analyst advises to buy a security. "Outperform" signals an analyst recommendation a stock is expected to do slightly better than the market return. Buy and outperform can also be expressed as "overweight", "moderate buy", "accumulate", "add", "buy" and "strong buy" advice.
	\end{minipage}
	} & \annexe{example}                                                                         \\ \thline
Security & Security being rated. & \\
Analyst & Rater, advisor or analyst publishing the rating. & \\
\specialrule{.1em}{.05em}{.05em} 
\end{tabularx}

\vspace{0.5cm}

\centering\begin{tabularx}{\textwidth}{| L{2.55cm} L{6cm} L{6.3cm} |}
\multicolumn{3}{C{14.85cm}}{\Large Rating\_\textbf{SellUnderperform}}                \\
\specialrule{.1em}{.05em}{.05em}
\multicolumn{2}{|L{8.55cm}}{
	\begin{minipage}{8.55cm}
		Analyst advises to sell a security or liquidate an asset. "Underperform" signals a recommendation that a stock is expected to do slightly worse than the overall stock market return. Sell or underperform can also be expressed as "moderate sell," "weak hold" and "underweight.". Includes "underweight", "underperform", "moderate sell", and "sell" and "strong sell" advice.
	\end{minipage}
	} & \annexe{example}                                                                         \\ \thline
Security & Security being rated. & \\
Analyst & Rater, advisor or analyst publishing the rating. & \\
\specialrule{.1em}{.05em}{.05em} 
\end{tabularx}

\vspace{0.5cm}

\centering\begin{tabularx}{\textwidth}{| L{2.55cm} L{6cm} L{6.3cm} |}
\multicolumn{3}{C{14.85cm}}{\Large Rating\_\textbf{Hold}}                \\
\specialrule{.1em}{.05em}{.05em} 
\multicolumn{2}{|L{8.55cm}}{
	\begin{minipage}{8.55cm}
	    Analyst advises to hold a security. A hold recommendation signals a security is expected to perform at the same pace as comparable companies or in-line with the market. Includes "hold" and "neutral" advice.
	\end{minipage}
	} & \annexe{example}                                                                         \\ \thline
Security & Security being rated. & \\
Analyst & Rater, advisor or analyst publishing the rating. & \\
\specialrule{.1em}{.05em}{.05em} 
\end{tabularx}

\vspace{0.5cm}

\centering\begin{tabularx}{\textwidth}{| L{2.55cm} L{6cm} L{6.3cm} |}

\multicolumn{3}{C{14.85cm}}{\Large Rating\_\textbf{Upgrade}}                \\
\specialrule{.1em}{.05em}{.05em} 
\multicolumn{2}{|L{8.55cm}}{
	\begin{minipage}{8.55cm}
		Analyst ratings, credit rating or advice are upgraded from an (implied) previous rating.
		For stocks, a "hold" advice can be upgrade to a "buy" advice.
		A "sell" advice can be upgraded to a "hold" or "buy" advice.
	\end{minipage}
	} & \annexe{example}                                                                         \\ \thline
Security & Security being rated. & \\
Analyst & Rater, advisor or analyst publishing the rating. & \\
HistoricalRating & The previous rating from which the security is upgraded. & \\
\specialrule{.1em}{.05em}{.05em} 
\end{tabularx}

\vspace{0.5cm}

\centering\begin{tabularx}{\textwidth}{| L{2.55cm} L{6cm} L{6.3cm} |}

\multicolumn{3}{C{14.85cm}}{\Large Rating\_\textbf{Downgrade}}                \\
\specialrule{.1em}{.05em}{.05em} 
\multicolumn{2}{|L{8.55cm}}{
	\begin{minipage}{8.55cm}
		Analyst ratings, credit rating or advice are upgraded from an (implied) previous rating.
		For stocks, a "buy" advice can be downgraded to a "hold" or a "sell" advice.
		A "hold" advice can be downgraded to a "sell" advice.
	\end{minipage}
	} & \annexe{example}                                                                         \\ \thline
Security & Security being rated. & \\
Analyst & Rater, advisor or analyst publishing out the rating. & \\
HistoricalRating & The previous rating from which the security is downgraded. & \\
\specialrule{.1em}{.05em}{.05em} 
\end{tabularx}

\vspace{0.5cm}

\centering\begin{tabularx}{\textwidth}{| L{2.55cm} L{6cm} L{6.3cm} |}

\multicolumn{3}{C{14.85cm}}{\Large Rating\_\textbf{Maintain}}                \\
\specialrule{.1em}{.05em}{.05em} 
\multicolumn{2}{|L{8.55cm}}{
	\begin{minipage}{8.55cm}
		Analyst security ratings, credit rating or advice remain in the same category from an (implied) previous rating.
	\end{minipage}
	} & \annexe{example}                                                                         \\ \thline
Security & Security being rated. & \\
Analyst & Rater, advisor or analyst publishing the rating. & \\
HistoricalRating & The previous rating from which the security remains stable. & \\
\specialrule{.1em}{.05em}{.05em} 
\end{tabularx}

\vspace{0.5cm}

\centering\begin{tabularx}{\textwidth}{| L{2.55cm} L{6cm} L{6.3cm} |}

\multicolumn{3}{C{14.85cm}}{\Large Rating\_\textbf{PriceTarget}}                \\
\specialrule{.1em}{.05em}{.05em} 
\multicolumn{2}{|L{8.55cm}}{
	\begin{minipage}{8.55cm}
		Analyst projects a price target for a security.
		A price target is the projected price level of a financial security stated by an investment analyst or advisor and includes assumptions of future activity.
		It represents a security's price that, if achieved, results in a trader recognizing the best possible outcome for his investment.
		This is the price at which the trader or investor wants to exit his existing position so he can realize the most reward.
	\end{minipage}
	} & \annexe{example}                                                                         \\ \thline
Security & Security being rated. & \\
Analyst & Rater, advisor or analyst publishing the rating. & \\
TargetPrice & The target price or value of a security as a range or singular price expressed as monetary value. & \\
\specialrule{.1em}{.05em}{.05em} 
\end{tabularx}

\vspace{0.5cm}

\centering\begin{tabularx}{\textwidth}{| L{2.55cm} L{6cm} L{6.3cm} |}

\multicolumn{3}{C{14.85cm}}{\Large Rating\_\textbf{Credit/Debt}}                \\
\specialrule{.1em}{.05em}{.05em} 
\multicolumn{2}{|L{8.55cm}}{
	\begin{minipage}{8.55cm}
		Analyst credit or debt rating for a company or organization.
		A corporate credit rating is an opinion of an independent agency regarding the likelihood that a corporation will fully meet its financial obligations as they come due.
		A company’s corporate credit rating indicates its relative ability to pay its creditors and gives investors an idea of how the company’s debt securities should be priced in term of yields.
		Credit assessment and evaluation for companies and governments is generally done by a credit rating agency such as Standard \& Poor’s (S\&P), Moody’s, or Fitch. 
	\end{minipage}
	} & \annexe{example}                                                                         \\ \thline
Security & Security being rated. & \\
Analyst & Rater, advisor or analyst publishing the rating. & \\
\specialrule{.1em}{.05em}{.05em} 
\end{tabularx}

\begin{itemize}[noitemsep,leftmargin=*]
	\item \url{https://www.investopedia.com/terms/c/corporate-credit-rating.asp}
	\item \url{https://www.investopedia.com/terms/c/creditrating.asp}
	\item \url{https://en.wikipedia.org/wiki/Bond_credit_rating}
\end{itemize}

\vspace{0.5cm}

%-------Revenue---------
\subsection{Revenue}

\centering\begin{tabularx}{\textwidth}{| L{2.55cm} L{6cm} L{6.3cm} |}
\multicolumn{3}{C{14.85cm}}{\Large \textbf{Revenue}}                \\
\specialrule{.1em}{.05em}{.05em} 
\multicolumn{2}{|L{8.55cm}}{
	\begin{minipage}{8.55cm}
	    Revenue is the total amount of income generated its diverse range of activities.
		Events regarding operating revenue is revenue generated from a company's primary business activities. 
		Non-operating revenue is revenue generated by activities outside of a company's primary operations; this revenue tends to be infrequent and oftentimes unusual.
		We include both revenue types.
		Revenue refers to gross and not net revenue.\\
		Overlap with \type{SalesVolume}: Sales are a part of the revenue and are considered almost identical to operating revenue in some industries.
		However, if explicit mention of sales volume or amount is made a SalesVolume event should be tagged.
	\end{minipage}
	} & \annexe{example}                                                                         \\ \thline
Company & Company responsible for the revenue. & \\
Amount & Amount of gross revenue expressed as monetary value. & \\
\specialrule{.1em}{.05em}{.05em}
\end{tabularx}

\begin{itemize}[noitemsep,leftmargin=*]
	\item \url{https://www.investopedia.com/terms/o/operating-revenue.asp}
	\item \url{https://en.wikipedia.org/wiki/Revenue}
	\item Sales Volume VS. Revenue:
	\begin{itemize}
	    \item Sales are the proceeds from the selling of goods or services to its customers.
        In accounting terms, sales make up one component of a company's revenue.
        On an income statement, sales are usually referred to as gross sales or "the top line" since sales are often used interchangeably with revenue.
    	It is important for businesses to closely track both sales and revenue. Keeping track of revenue is obviously important. As long as you keep expenses under control, increasing revenues means your business is likely to be flourishing.
    	By the same token, if your revenues are increasing but your net profits are not, that is a sure sign your expenses are up. Sales is the leading indicator for most companies, so most companies track sales closely.
    	\item \url{https://keydifferences.com/difference-between-sales-and-revenue.html}
    	\item \url{https://www.investopedia.com/ask/answers/122214/what-difference-between-revenue-and-sales.asp}
	\end{itemize}
\end{itemize}

\vspace{0.5cm}

\centering\begin{tabularx}{\textwidth}{| L{2.55cm} L{6cm} L{6.3cm} |}
\multicolumn{3}{C{14.85cm}}{\Large Revenue\_\textbf{Increase}}                \\
\specialrule{.1em}{.05em}{.05em} 
\multicolumn{2}{|L{8.55cm}}{
	\begin{minipage}{8.55cm}
        Revenue has increased compared to a historical trend.\\
		Overlap with \type{SalesVolume}: Sales are a part of the revenue and are considered almost identical to operating revenue in some industries.
		However, if explicit mention of sales volume or amount is made a SalesVolume event should be tagged.
	\end{minipage}
	} & \annexe{example}                                                                         \\ \thline
Company & Company responsible for the revenue. & \\
Amount & Current amount of gross revenue expressed as monetary value. & \\
HistoricalAmount & The historical revenue amount compared to which the current amount is higher. & \\
\specialrule{.1em}{.05em}{.05em} 
\end{tabularx}

\vspace{0.5cm}

\centering\begin{tabularx}{\textwidth}{| L{2.55cm} L{6cm} L{6.3cm} |}
\multicolumn{3}{C{14.85cm}}{\Large Revenue\_\textbf{Decrease}}                \\
\specialrule{.1em}{.05em}{.05em} 
\multicolumn{2}{|L{8.55cm}}{
	\begin{minipage}{8.55cm}
        Revenue has decreased compared to a historical trend.\\
		Overlap with \type{SalesVolume}: Sales are a part of the revenue and are considered almost identical to operating revenue in some industries.
		However, if explicit mention of sales volume or amount is made a SalesVolume event should be tagged.
	\end{minipage}
	} & \annexe{example}                                                                         \\ \thline
Company & Company responsible for the revenue. & \\
Amount & Current amount of gross revenue expressed as monetary value. & \\
HistoricalAmount & The historical revenue amount compared to which the current amount is lower. & \\
\specialrule{.1em}{.05em}{.05em} 
\end{tabularx}

\vspace{0.5cm}

\centering\begin{tabularx}{\textwidth}{| L{2.55cm} L{6cm} L{6.3cm} |}
\multicolumn{3}{C{14.85cm}}{\Large Revenue\_\textbf{Stable}}                \\
\specialrule{.1em}{.05em}{.05em} 
\multicolumn{2}{|L{8.55cm}}{
	\begin{minipage}{8.55cm}
        Revenue has remained stable or there is no significant change compared to a historical trend.\\
		Overlap with \type{SalesVolume}: Sales are a part of the revenue and are considered almost identical to operating revenue in some industries.
		However, if explicit mention of sales volume or amount is made a SalesVolume event should be tagged.
	\end{minipage}
	} & \annexe{example}                                                                         \\ \thline
Company & Company responsible for the revenue. & \\
Amount & Current amount of gross revenue expressed as monetary value. & \\
HistoricalAmount & The historical revenue amount compared to which the current amount has remained stable. & \\
\specialrule{.1em}{.05em}{.05em} 
\end{tabularx}

\vspace{0.5cm}

% \centering\begin{tabularx}{\textwidth}{| L{2.55cm} L{6cm} L{6.3cm} |}
% \thline
% \multicolumn{3}{C{14.85cm}}{\Large \textbf{Stock Holding}}                \\
% \specialrule{.1em}{.05em}{.05em} 
% \multicolumn{2}{|L{8.55cm}}{
% \begin{minipage}{8.55cm}
% Fund position, inside sell-purchase, index change.
% \end{minipage}
% } & \annexe{example}                                                                         \\ \thline
% \specialrule{.1em}{.05em}{.05em} 
% \end{tabularx}

\vspace{0.5cm}

%-------Sales Volume---------
\subsection{Sales Volume}

\centering\begin{tabularx}{\textwidth}{| L{2.55cm} L{6cm} L{6.3cm} |}
\multicolumn{3}{C{14.85cm}}{\Large \textbf{SalesVolume}}                \\
\specialrule{.1em}{.05em}{.05em} 
\multicolumn{2}{|L{8.55cm}}{
	\begin{minipage}{8.55cm}
		The quantity of goods and services sold over a certain period. This quantity can be expressed by a gross amount of money or . We include changes, announcements, declarations and forecasts of sales volume figures, increased, decreased, stable. Has conceptual overlap with Revenue and FinancialResult.
	\end{minipage}
	} & \annexe{example}                                                                         \\ \thline
GoodsService & The product, goods or services that makes up the sales volume & \\
Seller & The company, organization or entity selling the goods or services. & \\
Buyer & The buyer, group of buyers, or target market to which the change in sales volume can be attributed. & \\
Amount & Current amount of goods and services sold, expressed as monetary expression or as amount of units. & \\
\specialrule{.1em}{.05em}{.05em} 
\end{tabularx}

\begin{itemize}
    \item \url{https://www.accountingtools.com/articles/what-is-sales-volume.html}
\end{itemize}

\vspace{0.5cm}

\centering\begin{tabularx}{\textwidth}{| L{2.55cm} L{6cm} L{6.3cm} |}
\multicolumn{3}{C{14.85cm}}{\Large SalesVolume\_\textbf{Increase}}                \\
\specialrule{.1em}{.05em}{.05em} 
\multicolumn{2}{|L{8.55cm}}{
	\begin{minipage}{8.55cm}
		The quantity of goods and services sold over a certain period. We include changes, announcements, declarations and forecasts of sales volume figures, increased, decreased, stable. Has conceptual overlap with Revenue and FinancialResult.
	\end{minipage}
	} & \annexe{example}                                                                         \\ \thline
GoodsService & The product, goods or services that makes up the sales volume & \\
Seller & The company, organization or entity selling the goods or services. & \\
Buyer & The buyer, group of buyers, or target market to which the change in sales volume can be attributed. & \\
Amount & Current amount of goods and services sold, expressed as monetary expression or as amount of units. & \\
HistoricalAmount & The historical sales volume amount compared to which the current amount is higher. &  \\
\specialrule{.1em}{.05em}{.05em} 
\end{tabularx}

\vspace{0.5cm}

\centering\begin{tabularx}{\textwidth}{| L{2.55cm} L{6cm} L{6.3cm} |}
\multicolumn{3}{C{14.85cm}}{\Large SalesVolume\_\textbf{Decrease}}                \\
\specialrule{.1em}{.05em}{.05em} 
\multicolumn{2}{|L{8.55cm}}{
	\begin{minipage}{8.55cm}
		The quantity of goods and services sold over a certain period. We include changes, announcements, declarations and forecasts of sales volume figures, increased, decreased, stable. Has conceptual overlap with Revenue and FinancialResult.
	\end{minipage}
	} & \annexe{example}                                                                         \\ \thline
GoodsService & The product, goods or services that makes up the sales volume & \\
Seller & The company, organization or entity selling the goods or services. & \\
Buyer & The buyer, group of buyers, or target market to which the change in sales volume can be attributed. & \\
Amount & Current amount of goods and services sold, expressed as monetary expression or as amount of units. & \\
HistoricalAmount & The historical sales volume amount compared to which the current amount is lower. &  \\
\specialrule{.1em}{.05em}{.05em} 
\end{tabularx}

\vspace{0.5cm}

\centering\begin{tabularx}{\textwidth}{| L{2.55cm} L{6cm} L{6.3cm} |}
\multicolumn{3}{C{14.85cm}}{\Large SalesVolume\_\textbf{Stable}}                \\
\specialrule{.1em}{.05em}{.05em} 
\multicolumn{2}{|L{8.55cm}}{
	\begin{minipage}{8.55cm}
		The quantity of goods and services sold over a certain period. We include changes, announcements, declarations and forecasts of sales volume figures, increased, decreased, stable. Has conceptual overlap with Revenue and FinancialResult.
	\end{minipage}
	} & \annexe{example}                                                                         \\ \thline
GoodsService & The product, goods or services that makes up the sales volume. & \\
Seller & The company, organization or entity selling the goods or services. & \\
Buyer & The buyer, group of buyers, or target market to which the change in sales volume can be attributed. & \\
Amount & Current amount of goods and services sold expressed as monetary expression or as amount of units. & \\
HistoricalAmount & The historical sales volume amount compared to which the current amount is stable. &  \\
\specialrule{.1em}{.05em}{.05em} 
\end{tabularx}


%-------Security Value---------
\subsection{Security Value}

\centering\begin{tabularx}{\textwidth}{| L{2.55cm} L{6cm} L{6.3cm} |}
\multicolumn{3}{C{14.85cm}}{\Large \textbf{SecurityValue}}                \\
\specialrule{.1em}{.05em}{.05em} 
\multicolumn{2}{|L{8.55cm}}{
	\begin{minipage}{8.55cm}
		Events describing the value/price or change in value/price of a share, stock, derivative or any tradable financial asset. We also includes groupings of securities as in Exchange Traded Funds of market indices (This includes when descriptions and growth/decline/stability in a market index is discussed). We include announcements, forecasts, speculation, expectation, reports, etc. on price increase/decrease/stability or plain value/price descriptions that do not compare to a historical trend.
	\end{minipage}
	} & \annexe{example}                                                                         \\ \thline
Security & The security in question. & \\
Price & The current price of a single security. Expressed as monetary value. & \\
\specialrule{.1em}{.05em}{.05em} 
\end{tabularx}

\begin{itemize}[noitemsep,leftmargin=*]
	\item \url{https://www.investopedia.com/terms/s/security.asp}
	\item \url{https://en.wikipedia.org/wiki/Security_(finance)}
\end{itemize}

\vspace{0.5cm}

\centering\begin{tabularx}{\textwidth}{| L{2.55cm} L{6cm} L{6.3cm} |}
\multicolumn{3}{C{14.85cm}}{\Large SecurityValue\_\textbf{Increase}}                \\
\specialrule{.1em}{.05em}{.05em} 
\multicolumn{2}{|L{8.55cm}}{
	\begin{minipage}{8.55cm}
		Events describing an increase or positive percentage growth in value/price a share, stock, derivative or any tradable financial asset compared to a historical trend. We also includes groupings of securities as in Exchange Traded Funds of market indices (this includes when growth in a market index is discussed). Includes announcements, forecasts, speculation, expectation, reports, etc. on price increase.
	\end{minipage}
	} & \annexe{example}                                                                         \\ \thline
Security & The security in question. & \\
Price & The current price of a single security up to which the historical price increased. Expressed as monetary value. & \\
HistoricalPrice & The historical unit price of the security compared to which to current price is an increase. Expressed as monetary value. & \\
IncreaseAmount & Relative or nomimal price increase as monetary amount or percentage. & \\
\specialrule{.1em}{.05em}{.05em}
\end{tabularx}

\vspace{0.5cm}

\centering\begin{tabularx}{\textwidth}{| L{2.55cm} L{6cm} L{6.3cm} |}
\multicolumn{3}{C{14.85cm}}{\Large SecurityValue\_\textbf{Decrease}}                \\
\specialrule{.1em}{.05em}{.05em}
\multicolumn{2}{|L{8.55cm}}{
	\begin{minipage}{8.55cm}
		Events describing a decrease or negative percentage decline in value/price a share, stock, derivative or any tradable financial asset compared to a historical trend. We also includes groupings of securities as in Exchange Traded Funds of market indices (This includes when decline in a market index is discussed). Includes announcements, forecasts, speculation, expectation, reports, etc. on price reduction.
	\end{minipage}
	} & \annexe{example}                                                                         \\ \thline
Security & The security in question. & \\
Price & The current price of a single security. Expressed as monetary value. & \\
HistoricalPrice & The historical unit price of the security compared to which to current price is a decrease. Expressed as monetary value. & \\
DecreaseAmount & Relative or nomimal price decrease as monetary amount or percentage. & \\
\specialrule{.1em}{.05em}{.05em} 
\end{tabularx}

\vspace{0.5cm}

\centering\begin{tabularx}{\textwidth}{| L{2.55cm} L{6cm} L{6.3cm} |}
\multicolumn{3}{C{14.85cm}}{\Large SecurityValue\_\textbf{Stable}}                \\
\specialrule{.1em}{.05em}{.05em} 
\multicolumn{2}{|L{8.55cm}}{
	\begin{minipage}{8.55cm}
		Events describing stability or no significant percentage change in value/price a share, stock, derivative or any tradable financial asset compared to a historical trend. We also includes groupings of securities as in Exchange Traded Funds of market indices (This includes when decline in a market index is discussed). Includes announcements, forecasts, speculation, expectation, reports, etc. on price stability.
	\end{minipage}
	} & \annexe{example}                                                                         \\ \thline
Security & The security in question. & \\
Price & The current price of a single security up to which the historical price remained stable. Expressed as monetary value. & \\
HistoricalPrice & The historical unit price of the security compared to which to current price is stable. Expressed as monetary value. & \\
\specialrule{.1em}{.05em}{.05em}
\end{tabularx}

\vspace{0.5cm}

\section{FILLER arguments} \label{sec:FILLERtypes}

\justify
For information on when to annotate FILLER arguments, see \fullref{sec:FILLERargtagg}.
For the event ontology, we describe conceptual and taggability properties of individual FILLER arguments here.

\subsection{Universal FILLER Arguments: TIME, PLACE, CAPITAL}

\justify
Universal FILLER arguments are applicable to all events regardless of type and subtype.
Annotators should always be aware and looking for these FILLER arguments.
These arguments are universal because they are highly probable to be associated with any economic event:
An event is inherently temporal (\type{TIME}) and locational (\type{PLACE}): it occurs somewhere and takes up some point or duration in time.
In economic news, events often carry a cost or involve an amount of money. For this, we introduce the \type{CAPITAL} FILLER argument.

\subsubsection{TIME}

\justify
\noindent\textbf{On types and subtypes}: all.\\[6pt]
\noindent\textbf{Conceptually}:
A TIME is a temporal expression that corresponds to a specific duration and point in time.
This includes calendar dates (e.g., "October 29", "June 2017", "the 24th of April") but also relative time descriptions (e.g., "yesterday". "last week", "next year") and durations (e.g, "the coming months", "over five years").\\

\noindent\textbf{Taggability}:\\
The full nominal constituent NP of the duration of time point will be tagged.
This includes pre- and post-modifiers and excludes the preposition that is the head of the TIME noun phrase.

\begin{exe}
    \ex \annexe{In [ \exargfill{2017} ], the company announced their desire for entering the Chinese market.}
    \ex \annexe{The decision to replace the executive will be made [ \exargfill{next Thursday} ].}
    \ex \annexe{The leaders met [ \exargfill{last week} ] in Lagos.}
    \ex \annexe{They have been working on the deal for [ \exargfill{the better part of a year} ]. }
    \ex \annexe{The park will open [ \exargfill{Monday 23 June} ].}
    \ex \annexe{The Boeing Corp. announced the deal [ \exargfill{yesterday after Trump's visit to Iran} ].}
\end{exe}

\vspace{0.5cm}

\subsubsection{PLACE}

\justify
\noindent\textbf{On types and subtypes}: all.\\[6pt]
\noindent\textbf{Conceptually}:\\
PLACE encompasses specific mentions of geographic locations.
It refers to places (e.g., "Wall Street", "The White House", "Downing Street"), cities (e.g., "Washington", "Detroit"), countries ("US", ""), states (e.g., "Oregon", "Kentucky"), regions (e.g., "Central-Asia", "the Champagne region", "the fourth district", "CBD" (Central Business District)), and markets expressed as geographic locations (e.g., "the Indian market"). We also include nicknames (e.g., "the Big Apple", "Motor City").\\

\noindent\textbf{Taggability}:\\
The full NP containing the location expression is tagged.
This includes pre- and post-modifiers and not the preposition that is the head of the PLACE noun phrase.

\tagged{full}{
Rich ERE does not tag location FILLER arguments.
This information is particularly important for many economic events as business is legally bound by market and geography.)
}

\vspace{0.5cm}

\subsubsection{CAPITAL}

\justify
\noindent\textbf{On types and subtypes}: all.\\[6pt]
\noindent\textbf{Conceptually}:
Financial capital is any economic resource measured in terms of money used by entrepreneurs and businesses to buy what they need to make their products or to provide their services to the sector of the economy upon which their operation is based, i.e. retail, corporate, investment banking, etc.

A \type{CAPITAL} FILLER argument is the amount of money (or securities or commodities that hold transactional value) that is associated with an event. It is usually described in terms of the currency of some country or region (e.g., "US Dollars", "\$", "USD" or "Euros", "EUR"), but can also be expressed in an amount of securities, equity or debt (e.g., "1.5 million dollars in stock", "27,000 shares" or "3.7 million in Covered Bonds").\\

\noindent\textbf{\type{CAPITAL} vs. similar Participant arguments}:
Many event types have a Participant argument such as \type{Cost}, \type{Amount} or \type{Capital} that is conceptually similar to this FILLER argument.
However \type{CAPITAL} has other taggability rules w.r.t. Event Mention Scope - being a FILLER argument: 
Unlike the similar Participant arguments, \type{CAPITAL} can be tagged from anywhere in the document preferring the closest mention.
If a similar Participant argument is in Event Mention Scope tagging the argument \emph{always} takes precedence over tagging that mention as FILLER \type{CAPITAL}.
Furthermore, FILLER arguments are always nominal constituents and can never be pronominal unlike Participant arguments. \\

\noindent\textbf{Taggability}:
The extent of money a CAPITAL mention encompasses the full NP in which the capital amount is mentioned.
It needs to include modifying constituents as well as monetary units.
This includes quantifying adjective phrases (e.g., "almost", "a large amount of", "almost") and qualifying adjectives (e.g., "large", "exuberant", "plentiful").

\begin{exe}
    \ex \annexe{The company invested [ \exargfill{nearly 230 million dollars} ] in the newly founded lab.}
    \ex \annexe{Facebook faces fines of [ \exargfill{up to 11 billion Euros} ] if they do not comply with GDPR.}
    \ex \annexe{XYZ Corp. acquired [ \exargfill{18000 shares} ] in a 1:2 acquisition stock deal.}
    \ex \annexe{The position involves [ \exargfill{a plentiful \$~1,200,000 signing bonus} ].}
\end{exe}

\tagged{full}{
Similar to Rich ERE MONEY (MON) arguments, but in taggability resembles the COMMODITY (COM) filler argument. (cf., DEFT Rich ERE Annotation Guidelines: Argument Fillers v2.3 p11)
}

\vspace{0.5cm}

\subsection{Type-Specific FILLER Arguments}
These FILLER arguments are only tagged when they belong to a specific \type{type} or \type{subtype}.

\subsubsection{SENTENCE}

\justify
\noindent\textbf{On types and subtypes}: \type{Legal.Conviction/Settlement}.\\[6pt]
\noindent\textbf{Conceptually}:\\
A conviction is formal declaration by the verdict of a jury or the decision of a judge in a court of law that someone is guilty of a criminal offence.
A settlement is an official agreement intended to resolve a dispute or conflict.
Both are the outcome of a formal allegation in which one party is forced to a penalty by the court or the litigating party.
\\[6pt]
\noindent\textbf{Taggability}:\\
The full NP containing the sentence is tagged including any determiners, articles, pre- and post-modifiers.

\vspace{0.5cm}

\subsubsection{ALLEGATION}

\justify
\noindent\textbf{On types and subtypes}: any \type{Legal} type.\\[6pt]
\noindent\textbf{Conceptually}:
Allegation or other formal/legal offence that is alleged against a defendant.
ALLEGATION is tagged whenever an crime, misconduct or other offense is formally alleged against a company, organization or person.
\\[6pt]
\noindent\textbf{Taggability}:
The full NP containing the sentence is tagged including any determiners, articles, pre- and post-modifiers.

\vspace{0.5cm}

\subsubsection{TITLE}

\justify
\noindent\textbf{On types and subtypes}: any \type{Employment} type.\\[6pt]
\noindent\textbf{Conceptually}:
The job title/position of the an Employee or a certain professional role, type of worker.
Titles are the personal titles and honorifics, official rank or status, and specific employed occupations or professional positions. Annotators will need to use best judgment when determining whether a position or occupation is specific enough to be tagged as a title (e.g., general types such as worker, official, member, employee, will not be tagged TITLE).\\[6pt]

\noindent\textbf{Taggability}:\\
For our purposes, the extent of a TITLE is limited to the string of text that describes the position or rank itself, independent of its organizational circumstances or relationships, and excluding a preceding definite article (‘the’) and any other pre-posed or post-posed modifiers.

Because TITLEs occur in conjunction with other entities and refer to other entities, they have some special rules. The most frequent constructions TITLEs occur in are title + name, appositives, and copula constructions.

Just as with appositives, all titles, positions, and honorifics will be marked separately from the individual's name.

\tagged{full}{
\section{Event Type Conceptualization Notes}

\subsection{Continuous vs. Categorical change}
Many events in this typology express some form of change: either in rating, price, value, or some other observable result.
Among these events we can distinguish changes that apply to continuous values and categorical values.
A continuous event is a SalesVolume or SecurityPrice event.
A real price value changes gradually over time and the event structure captures a snapshot of this change.
A categorical change are Financial Result and Rating change. 
We expect the event trigger to inherently signal the change in value rather than a verb
as a predicate taking arguments that signal the change.

For instance, Rating\_Buy does not have a Participant argument Rating for its "buy" rating as the event type itself would be expressed by the mention of "buy".
The would-be Rating Participant argument is necessary for making the subtype distinction and therefore co-opted by the trigger extent.
This is unlike Continuous events which are more likely to be triggered by a verb that takes multiple semantic arguments: SecurityValue\_Increase inherently involves a historical price trend and a current price as Participant argument.
The process of increasing itself can easily be expressed by standalone verbs such as "increase", "grow", "go up".
}

% ISSUES:

% In SentiFM: you have subtypes based on attributes of the events: whether it is an announcement, cancellation, forecast, buy-rating/sell\_rating/hold\_rating etc. Some of these categories are
% => What to do with Forecasts and announcements: Boudoukh 2012 Sees this as a separate event and generalizes over the outcome of the forecast e.g. Guidance Change Growth (Guidance = forecast released by company on earnings/revenue; typically includes revenue estimates, along with earnings, margins and capital spending estimates; it is also known as "earnings guidance.")

% Decisions:
% Boudhoukh 2012 Has a category Investment
% Boudhoukh 2012 Financial: split up according to type
% Boudhoul 2012 Financing: 
% Boudhouk 2012: Forecast: