\section{Basic Concepts and Terminology}
\begin{description}[noitemsep]
    \item[Event] A description of a change in the state-of-affairs that involves prototypical participant arguments.
    \item[Mention] a span of words (tokens) corresponding to a specific annotation unit, i.e., event, Participant or Filler argument.
    \item[Extent] The textual boundaries (start and end) of a linguistic expression associated with an Participant or FILLER argument.
    The string of text we annotate to indicate a concept.
    The extent rules described in these guidelines define restrictions on linguistic constructions (e.g. Noun Phrase, Prepositional Phrase, Pronoun, Proper Noun, Noun, etc.) that are allowed for that category.
    Extent is different for Participant and FILLER argument.
    \item[Taggability] The rules and circumstances in which a running full-text mention is tagged.
    \item[Event trigger] The token (or multiple tokens) that evokes the event of a certain \type{Type\_Subtype}.
    \item[Event mention scope] The textual scope from which arguments and attributes are tagged for a specific event mention.
    The event mention scope definition specifies the start and end of a specific event mention in the document.
    \item[Event argument] Arguments describe the entities involved in an event. An event takes place and multiple persons, companies or organizations have certain prototypical roles in that event.
    \item[FILLER argument] A FILLER argument gives descriptive information about an event. \tagged{full}{This is the Rich ERE equivalent of "Argument FILLER".}
    \item[Participant argument] A participant argument is a prototypical participant role in an event. It captures who or what is involved in a specific relation to the event. \tagged{full}{This is the Rich ERE equivalent of "Event Participant Argument".}
    \item[Event mention] An instance of an event in full-text annotation.
    \item[Argument mention] An instance of an argument in full-text annotation.
    \item[Coreference] occurs when two or more expressions in a text refer to the same person or thing; they have the same referent. Two event or entity mentions are coreferent when they refer to the same event or entity conceptually.
\end{description}

\section{Typographical Conventions}
We lay out some typographical signifiers for consistency and ease of reading:
\\[10pt]
In \annexe{sentence examples}: \exargpart{Full underline} shows the event Participant arguments or potential Participant argument candidates.
\exargfill{Dashed underline} shows the FILLER arguments or candidate FILLER arguments.
When discussing mention extent [ \exargpart{square brackets} ] are used to highlight the correct delimitation of the mention span.
\anntrg{Bold} indicates the event trigger.
\\[10pt]
\textbf{Event tables} take the following form:\\[10pt]
\begin{tabular}{|l|l|} \hline
\type{Type.Subtype} of event & "event trigger token span" \\\hline
\type{EventParticipantArgument} in CamelCase & "Participant argument token span" \\
\type{FILLERARGUMENT} in ALLCAPS & "FILLER argument token span" \\\hline \end{tabular}
\\\\
e.g.,
\begin{exe}
\ex \annexe{The \exargpart{FDA} \anntrg{approved} \exargpart{the medicine} submitted by \exargpart{Johnson\&Johnson} in \exargfill{2013} and \exargpart{it} was \anntrg{launched} in \exargfill{February 2016}.}
    \expl \begin{tabular}{|L{6cm}|L{7cm}|} \hline
        \type{Product/Service\_Approval} & "approved" \\\hline
        \type{Approver} & "FDA" \\
        \type{Owner} & "Johnson\&Johnson" \\
        \type{ProductService} & "the medicine" \\
        \type{TIME} & "2013" \\
        \hline \end{tabular}
        \\\\ "approved" triggers \type{ProductService.Approval} event with \type{Owner} Participant argument "Johnson\&Johnson", \type{ProductService} Participant argument "product line", \type{Approver} Participant argument "FDA", and FILLER argument \type{TIME} "2013"
    \expl \begin{tabular}{|L{6cm}|L{7cm}|} \hline
        % \rowcolor{lightergray}
        \type{ProductService.Launch} & "launched" \\\hline
        \type{ProductService} & "it" \\
        \type{TIME} & "February 2016" \\\hline \end{tabular}
        \\\\ "launched" triggers event \type{ProductService.Launch} \type{ProductService} Participant argument "it" and \type{TIME} FILLER argument "February 2016".
\end{exe}
